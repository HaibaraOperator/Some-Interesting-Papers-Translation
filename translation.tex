\documentclass{article}
\textheight 21cm % 每页高度
\textwidth 14cm % 每页宽度
\usepackage{amsfonts}
\usepackage{arcs}
\usepackage{bm}
\usepackage{ctex}
\usepackage{empheq} % 公式加框
\usepackage{epsfig}
\usepackage{amsmath}
\usepackage{graphicx}
\usepackage{feynmp}
\usepackage{indentfirst} % 开头不顶格
\usepackage{slashed} % Feynman斜线
%\usepackage{txfonts} % 可以用oiint积分
\newcommand{\p}{\partial}
\newcommand{\n}{\nonumber}
\newcommand{\nb}{\numberwithin{equation}{section}}
\title{\textbf{量子引力为何遥不可及 (相较于 QCD)?}}
\author{Hidenori Fukaya\footnote{原作者为日本大阪大学物理系副教授, 原文为日本文献 (源文献发表于《素粒子論研究》杂志 2016 年 25 卷第 2 册) 在 arXiv 上的英文版本, arXiv 号: 1811.11577. 译注: 为统一特殊名词, 译文中所有人名及期刊名均统一采用英文形式.}}
\date{翻译: 未婚妻姓逢田, 詹开. 校对: 未婚妻姓逢田.}
\begin{document}
\CJKspace
\newpage
\pagenumbering{Roman}
\newpage
\pagenumbering{arabic}
\maketitle


%======================问题介绍====================================
\begin{center}
\section*{摘要}
\end{center}
众所周知, 想要将引力量子化是极为困难的一件事. 以往我们总是将其原因简单地归结于 Newton 常数的不可重整性, 而很少谈及为什么在众多量子规范理论中唯有引力如此特殊. 在本文中, 我们尝试将引力视为一种规范理论, 从而讨论它的特殊性体现在何处, 以及为何量子引力如此遥不可及.
\section{引言}
本文作者主要从事格点 QCD 的数值模拟工作. 这个领域的基本研究对象是处于 $\sim$GeV 级别能标的夸克与胶子组成的多体系统, 而在这样的系统中引力效应可以说是微不足道的, 所以在这之前作者对于广义相对论也并不熟悉. 但在 2014 年, 作者被要求在大四学生的研讨课程上讲授广义相对论, 这使得他不得不花费了大量时间, 重新拾起许多已经忘却的知识. 但实际上, 由于那一年出现了 80 年来从未出现过的情况——没有任何学生选择高能理论的课题组, 研讨课还未开始便已经结束了.
\par
已然没有了授课的压力, 作者本可以在此停止引力理论的学习并再一次地将它们抛在脑后. 但他仍然把每周的这一个半小时当做宝贵的学习机会, 他将引力与他所熟悉的 QCD 进行比较并试图理解引力的不同究竟在何处, 以及为何引力的量子化会如此困难. 引力量子化的举步维艰是公认的事实, 通常人们也认为源自于 Newton 常数负质量量纲的不可重整性是对此很好的解释. 然而很少有教材会提及为何在众多规范理论中唯有引力如此特殊并存在许多问题. 经过一个学期的研究, 作者得到了一些结论并在他的课题组内进行了发表. 他的同事 Kinya Oda 在听完发表后, 提出了基于这些结论撰写一篇文章并提交至 Soryushiron Kenkyu 杂志\footnote{即开头提到的素粒子論研究杂志, 为日本高能粒子理论方向电子期刊}的想法, 这便是这篇文章诞生的来由.
\par
广义相对论是考虑了广义协变性 (以及局域 Lorentz 对称性) 的一种规范理论, 而量子色动力学 (QCD) 则是考虑 $SU(3)$ 色不变性的规范理论. 这两种理论都服从规范原理, 即二者在局域规范变换下均是不变的, 从而它们在经典层面上会具有相似的性质. 比方说它们的拉氏量都以由联络规范场定义的曲率张量来表示.
\par
但它们也存在一些不同之处. 在 QCD 中, 规范势或联络 $A_{\mu}$ 是理论的基本自由度 (d.o.f.), 但在广义相对论中, 联络不再是基本自由度, 而是由度规张量 $g_{\mu\nu}$ 来描述其中的物理. 且它们在运动方程 (EOM) 上存在极大的差异, 对于 QCD, EOM 的解往往是无时间演化的静态或驻点解, 而广义相对论的运动方程解却一般是动力学的, 且具有很强时间关联的解, 比如不考虑宇宙学常数精细调节时的暴涨. 同时在 QCD 中, 我们无法找到可以和广义相对论中标架场 (vierbein) 与挠率张量相对应的项. 综上所述, QCD 是可重整的, 而引力至少在微扰论的观点下是不可重整的. 二者间的相似与差异如表 1 所示.
\begin{table}
\begin{center}\caption{引力与 QCD 的相似与差异}
\begin{tabular}{|c|c|c|}
  \hline
  % after \\: \hline or \cline{col1-col2} \cline{col3-col4} ...
   & 广义相对论引力 & QCD \\
   \hline
  联络 & $\Gamma^{\rho}_{\ \mu\nu}$ & $A_{\mu}$ \\
  曲率 & $R^{\sigma}_{\ \mu\nu\rho}$ & $F_{\mu\nu}$ \\
  基本自由度 & $g_{\mu\nu}$ & $A_{\mu}$ \\
  拉氏量 & $\sqrt{g}R$ & $\textrm{Tr}{F_{\mu\nu}F^{\mu\nu}}$ \\
  运动方程解是否为静态或驻点解? & 否. & 是. \\
  是否存在标架与挠率? & 是. & 否 \\
  可重整性 & 不可重整! & 可重整! \\
  \hline
\end{tabular}

\end{center}
\end{table}
\indent 作者是从 Utiyama [2] 的教材开始学习的, 这是大阪大学最优秀的教授之一. 这本教材很清楚地写下了 Utiyama 在早年曾经做过的为引力与 Yang--Mills 理论给出统一理解的尝试. 在书中 Utiyama 提及, 他最初的尝试实际上与数学中的纤维丛构造异曲同工, 但直到后来他才意识到这点, 因为在当时纤维丛与统一场论一样只是一个新兴的还在发展的理论. 紧接着, 作者又阅读了 Nash 与 Sen [3] 的教程. 这本书则明确指出 Riemann 流形就是联络由度规给出的一类特殊的纤维丛. 而在 Kobayashi 与 Nomizu [4] 的书中, 他们解释了 Riemann 流形为何特殊以及它特殊在何处. 在将这些数学上的事实翻译为物理语言后, 我们便能够解释广义相对论与规范理论之间的差异了. 而这实际上就是 1--形式的广义相对论 (或称 Palatini 形式) 所考虑的, 在 Palatini 形式中我们视联络与标架为基本场, 度规则不再是基本自由度了.
\par
在这篇文章中, 我们将通过 1--形式广义相对论与其他规范理论的比较, 以纤维丛语言讨论引力的特殊性质是如何又为何出现的. 然后我们便能得出量子化引力如此困难的原因 (包括不可重整性), 可以完全归结为一句话:
\begin{center}
\textbf{标架丛是可平行化的.}
\end{center}
标架丛是一种广义的 Riemann 流形, 它们之间可视作父与子的关系.
\par
除了一些作者的评论外, 本文的内容并不是原创结果, 而是在数学上或物理上已经为学界所接受的认知. 前述结论亦不是原创结论, 在文献 [8] 中 Heller 早已在 2006 年提出了此结论. 本文的主要目的, 是在 Newton 常数不可重整这样的显然事实背后的深邃之中, 为读者攫取一丝异样的光芒.
\section{纤维丛}
市面上有不少关于纤维丛的教材, 其中也有一些是专门为物理学家编写的[3,9]. 这些教材之中, 我们主要参考的是由 Nash & Sen 编写的 [3]. 该书在第七部分纤维丛引入之前, 并不涉及任何关于 Riemann 流形的内容, 所以对于物理学家而言是极其友好的. 这里我们不会重新叙述纤维丛的精确数学定义, 而是使用物理语言, 特别是高能粒子理论的语言, 简要地给出大致梗概.

纤维丛是一类复合流形 $E$, 它由表示时空的底流形 $M$ 以及表示场空间的纤维空间 $F$ 组成. 在物理上(四维情形), 场即时空中的各个点, 并由直积 $\mathbb{R}^{4}\times F$ 定域地表示. 如果对此直积进行全局地延展, 那么纤维丛 $E$ 的总空间就是 $ M\times F$, 尽管仅看这一点并不是很有趣, 但在一般情形下纤维丛的全空间往往具有非平凡的结构.

在纤维丛的定义中存在结构群 $G$, 它可以生成纤维空间 $F$ 中的坐标变换. 举个例子, 如果我们取 $F$ 为一复平面, 而 $G$ 为 $U(1)$ 群, 那么 $\phi(x) \in F$ 的坐标变换由 $g(x) \in G, x \in M$ 给出:
\begin{equation}
\phi'(x)=g(x)\phi(x).
\end{equation}
从这个例子中可清晰地看到, 纤维中的坐标变换就是物理学家所说的规范变换. 注意尽管在很多教材中都不会强调 $G$ 只能线性地作用在纤维空间这一点, 但我们之后会提到, 这个条件实际上是非常重要的.

现在定义一类通过坐标变换 (在基和纤维方向) 相联系的纤维丛. 它的非平凡结构源于覆盖 $M$ 的不同开区域 $U_{i}$ 的坐标 (转变函数) 之间的连续性条件. 在物理中, 瞬子位形就是一个典型的非平凡纤维丛, 考虑四维球面为底流形 $M = S^{4}$, 纤维空间服从 $G = SU(2)$ 变换, 这时我们分别需要两个包含南北两极 $S^{4}$ 的小块, 两小块的重叠区域可标记为 $S^{3}$. 两组``坐标''通过规范变换相联系, 使得存在映射 $S^{3}\rightarrow G$. 若此映射是非平凡的, 我们就无法将总纤维丛表示为直积 $E = M\times F$ 的形式.

\par 也可以令结构群 $G$ 本身作为纤维空间, 即 $F = G$. 在这种情况下, 该纤维丛称为主丛, 通常由 $P$ 表示. 从 $P$ 出发,我们可以构造带有纤维 $F_{G}$ 的纤维丛, 它取自被称作配丛 $G$ 的任意表示空间. 配丛由商群 $E = P\times F_{G}/G$ 决定. 该主丛及其配丛的构造对应于物理学中的规范原理: 时空 $M$ 和规范群 $G$ 足以构造一个自洽的量子场论, 该场论中不同表示的场则以配矢量空间的形式引入.

\par 我们可以通过所谓的联络为主丛提供一个局域结构. $P$ 的联络由点 $u \in P$ 处的切空间分解得到: $T_{u}(P)$ 分为底空间中一个垂直子空间 (或平行纤维空间) $V_{u}(P)$ 和一个平行子空间 $H_{u}(P)$ (或垂直纤维空间). 这种分解可以在 $u\in P$ 的任一点得到, 且会使 $P$ 的总切丛变为 $T(P)= V(P)\oplus H(P)$\footnote{该联络由切丛 $T(P)$ 而不是主丛 $P$ 给出 (底空间为 $P$). 丛的概念对于像笔者一般的初学者而言是非常难以直观理解的.}.

\par 更具体地说, $P$ 的联络是通过一种被称作 1--形式联络的微分形式得到. 将 $P$ 上的局域坐标表示为 $u =(x, g)[x\in \mathbb{R}^{4}, g\in G]$, 那么该 1--形式联络为:
\begin{align}
\omega = g^{-1}dg+g^{-1}Ag, A=A_{\mu}^{a}(x)T_{a}dx^{\mu},
\end{align}
其中, $A_{\mu}^{a}(x)$ 是我们所熟知的矢势, $T_{a}$ 是群 $G$ 的生成元. 要实现 $T(P)$ 的分解, 必须要求任一矢量 $X \in H(P)$ 满足:
\begin{equation}
\langle \omega, X\rangle=0.
\end{equation}
这里, $\omega$ 的自由度与 $G$ 的维数一致. 在某种意义上, $\omega$ 就相当于 $H_{u}(P)$ 的法矢量.

\par 只要我们要求规范场变换为:
\begin{align}
A \rightarrow hdh^{-1}+hAh^{-1}.
\end{align}
那么 $\omega$ 在纤维方向的坐标变换 $g \rightarrow hg$ 下就是保持不变的. 这正是规范变换的体现. 由 1--形式联络 $\omega$ ,我们可定义 2--形式曲率 $\Omega$:
\begin{equation}
\Omega=d\omega+\omega\wedge \omega=g^{-1}(dA+A\wedge A)g=g^{-1}Fg,
\end{equation}
其中 $F$ 为场强.

\par 如果你不能很好的理解纤维丛的数学, 我们再提一种不太严谨的简单类比. 假设你的脑袋为底流形 $M$, 那么纤维 $F$ 就是你的头发, 纤维丛 $E$ 的总空间就是你的发型. 我们仍未知道那天你的杀马特中存在着的非平凡拓扑结构的名字.
\section{纤维丛观点下的 QCD}
在本节中, 我们将讨论如何利用纤维丛来描述 QCD. 这里取底流形 $M$ 为四维平坦 Euclid 空间而规范群为 $SU(3)$, 这对一位如笔者一样的格点高能理论研究者而言, 可以说是再熟悉不过的了. 而这些足以定义主丛并引入它们的联络. 注意到目前为止, 我们还不曾给出度规.

\par 那么量子场论应如何用该机制描述? 考虑这样一种情况, 具有给定联络 $A$ 的纤维空间 $P$ 以一种概率 $\rho$ 的形式出现. 这种统计诠释对应物理中的泛函积分. 假定 $\rho$ 为一标量, 不包含度规的 (几乎) 唯一的标量为:
\begin{equation}
S_{\theta}=\frac{\theta}{4}\int_{M}\,\textrm{Tr}F\wedge F,
\end{equation}
这是与 $\theta$ 有关的作用量. 同时可以自然地将 $\rho$ 写为 $\rho=\exp(iS_{\theta})$ (仅相差一常数乘子). 注意到
 $S_{\theta}$ 并不需要度规的存在\footnote{$\rho=\exp(iS_{\theta})$ 的理论对应一类拓扑场论.}, 这是一件很有趣的事,
 因为写出一个无度规的不可重整化作用量是不可能的. 但是在 QCD 中由于 $S_{\theta}$ 非常小, 我们通常都会直接忽略它的影响.
\par 当然,为底流形 $M$ 赋予度规 $g_{\mu\nu}=\textrm{diag}(1,1,1,1)$ 仍然是必要的. 这允许我们定义 2--形式曲率的 Hodge 对偶,
\begin{equation}
\star F_{\mu\nu}=\frac{1}{2}F_{\alpha\beta}g^{\alpha\gamma}g^{\beta\delta}\epsilon_{\gamma\delta\mu\nu}.
\end{equation}
从而可以构造出通常的规范作用量:
\begin{equation}
S_{g}=\frac{1}{4g^{2}}\int_{M}\,\textrm{Tr}F\wedge *F+ \cdots,
\end{equation}
这里 $\cdots$ 代表了包含不可重整项的无穷多种作用量.
\par 接着引入夸克场. 如上所述, 可由主丛 $P$ 构造配矢量丛 $Q$ :
\begin{equation}
Q=P\times F/G.
\end{equation}
对于夸克场, 可以选取 $F$ 为 $SU(3)$ 的基础表示. 利用 Dirac 算符 $D$, 可由一规范不变标量定义其作用量:
\begin{equation}
S_{q}=\int_{M}\,d^{4}x\sqrt{g}\bar{q}Dq.
\end{equation}
可以发现纤维 $F$ (的截面) 是显含于 $S_{q}$ 中的, 但在 $S_{g}$ 中却没有规范自由度 $G$ 的出现.

\par 因此, QCD 可由统计力学中的概率 $\rho= exp(-S_g +iS_\theta-S_q)$ 来描述, 其中主丛 $P$ 与配丛 $Q$ 均可自由变形 (freely deformed). $\rho$ 可以写成指数形式可能与作用量的广延性质或集团分解原理有关, 但不在此赘述其细节.

\par 在本节的最后想要指出一点, 格点规范理论与纤维丛实际上是具有相似结构的. 虽然格点理论是在离散化的格点时空中定义的, 但其结构群仍然是连续的. 因此主丛的纤维空间亦是连续的. 为格点赋予规范自由度相当于给出纤维空间 $G$, 从而可以确定其主丛结构. 格点间相联系的变量, 亦称作连接变量, 则给出了点与点间的``联络''. 在对给定区块的振幅进行限制从而将格点曲率光滑化后, 即可定义瞬子数. 在保持区块内光滑条件时, 瞬子数对连接变量的变分是保持不变的[10]. 事实上, 即使是``差分形式''及其上同调也可以在格点上良好定义[11], 这在格点手征规范理论的构造中是极其重要的.
\section{纤维丛观点下的引力与 1--形式焊接}
现在让我们用纤维丛的观点来讨论 (1--形式的) 广义相对论. 考虑一个四维流形 $M$, 现在暂且不给定度规并令一般线性群 $G=GL(4,\mathbb{R})$ 作为该理论的规范群. 我们由此定义了主丛 $F(M)$, 亦称作标架丛.
\par
标架丛拥有一类其他一般丛并不具有的性质——标架丛是可平行化的, 或者说, $T(F(M))$ 总是平凡的. 可平行化流形是一种可以在其中人为定义一个全局切矢空间的流形.
\par
我们以二维流形为例, 如果此流形是可平行化的, 那么我们可以在全局范围定义南北以及东西方向. 但在二维球面上这是不可能的, 总会存在两个奇点: 北极与南极. 而在环面上, 画出一系列环绕整个环面的平行线就是可能的了, 因为它可以看做是对边粘接起来的平行四边形.
\par
在拥有四维基底的流形 $M$ 上的标架丛是一个 $4+4^{2}=20$ 维流形, 尽管凭人力很难想象如此高维度中的可平行性, 但我们可以用如下方式证明它.
\par
切矢空间 $T_{x}M$ ($x\in M$) 是 $\mathbb{R}^{4}$, 主丛 $GL(4,\mathbb{R})$ 基础表示下的配丛亦然. 对于 $v\in V$ 与 $t\in T_{x}M$, 它们之间存在一一映射 $e$ 满足
\begin{align}
v^{a}=e^{a}_{\mu}t^{\mu}.
\end{align}
$e=(e_{\mu}^{1}dx^{\mu},e_{\mu}^{2}dx^{\mu},e_{\mu}^{3}dx^{\mu},e_{\mu}^{4}dx^{\mu})$ 是具有四分量的 1--形式, 亦称做 1--形式焊接. 焊接形式并不特殊, 它就是物理中标架场的表示. 在标架丛中, 配丛与底流形上的切空间是同构的, 这可以自然地给出焊接形式.
\par
确切地说, $e$ 不应该被称做焊接形式. 虽然 $e$ 的确是 $M$ 上的 1--形式, 但在坐标变换时 $e$ 并不是不变的. 与坐标无关的焊接形式 $\theta$ 应由 $F(M)$ 中的 1--形式来定义, 其形式为
\begin{align}
\theta=g^{-1}e.
\end{align}
注意在规范变换 $g\rightarrow hg$ 下, $e$ 的变换方式为 $e\rightarrow he$, 从而 $\theta$ 是保持不变的. 这里我们采用 $F(M)$ 的局部坐标\footnote{(12) 式的导出是建立在特定坐标 $u=(x,g)$ 的选取之上的. 想要得到与坐标基底选取无关的定义, 需使 $\theta$ 对任意 $X\in T(F(M))$ 满足
\begin{align}
\langle\theta,X\rangle=\langle e,\pi_{*}(X)\rangle.
\end{align}
这里 $\pi_{*}$ 为射影 (表示 $F(M)$ 中的某点在产生纤维的 $M$ 中对应哪一点) $\pi:F(M)\rightarrow M$ 的诱导映射 $\pi_{*}:T(F(M))\rightarrow T(M)$. 这时 $e$ 便是 $\theta$ 的一个拉回映射. 此定义看起来比 (12) 式要难理解得多, 但我们能更清晰地看到 $\theta$ 是定义在 $F(M)$ 上的.} $u=(x,g)$.

焊接形式 $\theta$ 为 $F(M)$ 的 1--形式, 但只有 $x$ 方向上存在非零分量. 如果某个位于 $x$ 点的切矢量 $X$ 恰好在纤维方向上或者 $X\in V_{u}(F(M))$, 那么可以证明 $\langle\theta,X\rangle=0$\footnote{在不选取坐标的定义下, 该式可以由 $\pi_{*}(X)=0(X\in V_{u}(F(M)))$ 证明.}. 如式 (3) 所示, 1--形式联络与 $H_{u}(F(M))$ 中任意矢量的内积为零. 从而, 对于任意 $X\in V_{u}(F(M))$, 可以得到
\begin{equation}
\langle\omega, X\rangle=0\&\langle \theta, X\rangle=0\Leftrightarrow X=0.
\end{equation}
这意味着 $X\neq 0$ 对应至少存在一类非零内积, 可以是与 $\omega$ 的也可以是与 $\theta$ 的, 即任意 $X$ 均可取 $w$ 与 $\theta$ 为 (对偶) 基底构造得到. 实际上, $\omega$ 有 $4\times4=16$ 个自由度而 $\theta$ 有 4 个, 它们的自由度总和 20 恰好对应 $F(M)$ 的 (切空间) 维度. 由于 $\omega$, $\theta$ 均是光滑地定义在 $F(M)$ 上的, 所以任意切矢量场 $X$ 在 $F(M)$ 上也同样是光滑的, 故而 $F(M)$ 是可平行化的.

在这种情况下, 由于 $F(M)$ 的可平行性我们不仅可以定义联络形式, 也可以定义焊接形式, 这告诉了我们为什么引力如此特殊: 引力理论不仅需要规范场而且需要标架场作为基本自由度. 我们接着引入 2--形式挠率,
\begin{equation}
\Theta=d\theta+\omega\wedge \theta.
\end{equation}
可以清楚地看到“挠率”在标架丛 $F(M)$ 中的表示是特别的.

在为引力理论准备好了坚实的数学基础之后, 让我们来计算自由度. 作为 $GL(4,\mathbb{R})$ 的生成元之一, 规范联络场有 $4^{2}\times4=64$ 个自由度, 而标架场有 $4\times4=16$ 个. 一共有 80 个自由度, 这看起来远多于我们预想的仅有两种物理模式的引力子的自由度, 但接下来的部分, 我们会看到这众多自由度是如何由不同物理条件消去的.

首先, 我们需要标架丛 $F(M)$ 的约化形式, 一般线性群可以写成  $GL(4,\mathbb{R})=O(4)\times C$, 这里 $C$ 为一个可以光滑收缩为一点的分量. 为了消除 $C$ 的影响并为主丛给出约化结构群 $O(4)$ 的操作称为主丛约化. $O(4)$ 对应 (Euclid 几何中的) 局域 Lorentz 群. 此约化允许我们按照如下方式定义 Riemann 度规:
\begin{equation}
g_{\mu\nu}=e_{\mu}^{a}e^{b}_{\nu}\eta_{ab}, \quad\eta_{ab}=\textrm{diag}(1,1,1,1)
\end{equation}
这里由于 $\eta_{ab}$ 是 $O(4)$ 下的不变张量, 我们也可以引入仿射联络
\begin{equation}
\Gamma_{\ \mu\nu}^{\lambda}=[A_{\nu}^{A}]_{b}^{a}\eta_{ca}e_{\mu}^{b}e_{\sigma}^{c}g^{\sigma\lambda}+(\textrm{微分项}).
\end{equation}
微分项的细节会在稍后讨论, 在此式中, $A_{\nu}^{A}$ 为约化 $O(4)$ 规范场. 值得注意的是 $O(4)$ 的指标 $a$, $b$ 是完全缩并的, 从而 $g_{\mu\nu}$ 与 $\Gamma_{\ \mu\nu}^{\lambda}$ 均为 $O(4)$ 不变的. 在 QCD 中, 由于不存在标架场能用来实现缩并, 自然也就不会出现类似的现象.

 在这种情况下, 广义协变性以规范对称性的涌现形式存在, 我们都知道广义协变性是在``局域''平移下保持两矢量场内积
 \begin{equation}
 g_{\mu\nu}(x)X^{\mu}(x)Y^{\nu}(x)
 \end{equation}不变, 这由 $g_{\mu\nu}$ 的条件 (度规性条件 (metricity condition))
\begin{equation}
\nabla_{\rho}g_{\mu\nu}\equiv \frac{\partial g_{\mu\nu}}{\partial x^{\rho}}-g_{\mu\sigma}\Gamma_{\ \nu\rho}^{\sigma}-g_{\nu\sigma}\Gamma_{\ \mu\rho}^{\sigma}=0
\end{equation}
实现. 很有趣的一点是广义协变性作为广义相对论的基本性质, 却以不变性 (或二阶不变性) 的涌现形式出现. 另外, 如果不考虑旋量场, 原本的 $GL(4,\mathbb{R})$ 或 $O(4)$ 规范对称性便完全成为隐对称性了.

在计算自由度的过程中读者也许会有些困惑. 在对标架丛进行约化后, $O(4)$ 规范场有 $6\times4=24$ 个自由度, $g_{\mu\nu}$ 有 10 个, 而保持 $g_{\mu\nu}$ 不变的标架场有 6 个, 即总共有 40 个. 但式 (19) 中的度规性条件显然已经给出了 40 个约束, 如果这两类约束相互独立, 物理自由度将变为 $80-40-40=0$. 而实际上, 约化后仍然应当存在 40 个自由度, 这意味着标架丛约化与度规性条件的约束是等价的.

标架丛约化与度规性条件的等价性鲜有文献提及. 它们的等价性要求存在一些方程在实现标架丛约化的同时也满足度规性条件, 实际上标架场的``运动方程''
\begin{equation}
[D_{\nu}e_{\mu}]^{a}=(\partial_{\nu}\delta_{b}^{a}+[A_{\nu}]^{a}_{b}e_{\mu}^{b})=0
\end{equation}
正是我们需要的方程. $A_{\nu}$ 为原有的 $GL(4,\mathbb{R})$ 规范场, $D_{\nu}$ 则表示协变导数.

我们将规范场分解为对称部分 $A_{\nu}^{S}\ (10*4=40\ d.o.f.)$ 与反对称部分 $A_{\nu}^{A}\ (6*4=24\ d.o.f.)$: $A_{\nu}=A_{\nu}^{S}+A_{\nu}^{A}$. 那么上述运动方程变为
\begin{equation}
(\partial_{\nu}\delta_{b}^{a}+[A_{\nu}]^{a}_{b})e_{\mu}^{b}=-[A_{\nu}^{S}]_{b}^{a}e_{\mu}^{b}
\end{equation}
等式左侧共有 40 个自由度 ($e_{\mu}^{a}(16),A_{\nu}^{A}(24)$), 而右侧在 $A_{\nu}^{S}$ 中有 40 个自由度. 此方程将 $A^{S}$ 看做 $e_{\mu}^{a}$ 与 $A_{\nu}^{A}$ 的函数, 这时它是约化 $O(4)$ 群的生成元, 同时 $A^{S}$ 可以在从 $GL(4,\mathbb{R})$ 到 $O(4)$ 的丛变换中实现标架丛约化.

进一步的, 如果定义
\begin{equation}
\Gamma_{\ \mu\nu}^{\rho}=-[A^{S}_{\nu}]^{a}_{b}e_{\mu}^{b}[e^{-1}]^{\rho}_{a}
\end{equation}
那么式 (21) 将改写为,
\begin{equation}
[\bar{D}_{\nu}e_{\mu}]^{a}\equiv (\partial_{\nu}\delta_{b}^{a}+[A_{\nu}^{A}]^{a}_{b})e_{\nu}^{b}=\Gamma^{\rho}_{\ \mu\nu}e_{\rho}^{a}
\end{equation}
该方程称为标架场假设. 这里 $\bar{D}_{\nu}$ 为服从 $O(4)$ 规范变换的协变导数. 由这条假设方程可以自动给出度规性条件,
\begin{equation}
\frac{\partial }{\partial x_{rho}}(g_{\mu\nu})=\frac{\partial }{\partial x_{rho}}(e_{\mu}^{a}e_{\nu}^{b}\eta_{ab})=[\bar{D}_{\rho}e_{\nu}]^{a}e_{\nu}^{b}\eta_{ab}=\Gamma^{\lambda}_{\ \mu\rho}g_{\lambda\nu}+\Gamma^{\lambda}_{\ \nu\rho}g_{\lambda\mu}
\end{equation}
从而我们由 (20) 式确认了在 $GL(4,\mathbb{R})$ 标架丛约化至 $O(4)$ 丛的过程中度规满足式 (19). 故剩余自由度为 40. 仿射联络由下式给出,
\begin{equation}
\Gamma_{\ \mu\nu}^{\lambda}=[e^{-1}]^{\lambda}_{a}[\bar{D}_{\nu}e_{\mu}]^{a}=[A_{\nu}^{A}]_{b}^{a}\eta_{ca}e_{\mu}^{b}e_{\sigma}^{c}g^{\sigma\lambda}+(\partial_{\nu}e_{\mu}^{a})\eta_{ca}e_{\sigma}^{c}g^{\sigma\lambda}.
\end{equation}
现在我们需要引入引力的另一条重要原理——等效原理, 由等效原理, 我们可以通过坐标变换使得仿射联络在局部为零. 其充要条件为 $\Gamma_{\ \mu\nu}^{\lambda}=\Gamma_{\ \nu\mu}^{\lambda}$, 或者等价的挠率 $T_{\mu\nu}=\Gamma_{\ \mu\nu}^{\lambda}-\Gamma_{\ \nu\mu}^{\lambda}$ 为零. 由 (23) 式, 此条件可以表示为
\begin{equation}
[\bar{D}_{\nu}e_{\mu}]^{a}-[\bar{D}_{\mu}e_{\nu}]^{a}=0
\end{equation}
由于上述方程为 2--形式, 等效原理也将给出 $\Theta=0$ 的结果. 挠率具有 24 个自由度, 与 $O(4)$ 规范场的自由度一致. 从而我们能够在理论上完全消除 $O(4)$ 的规范场 $A_{\mu}$, 即理论中仅需标架场的 16 个自由度便可完整地描述. 这样的理论看起来与 Yang-Mills 理论全然不同. 我们自然也不再奇怪为何引力在经典层面就已经如此难以处理.

固定局域 Lorentz 规范 (这样消去了 6 个自由度), 我们得到广义相对论的一般形式 (2--形式), 在此形式中我们仅需要度规 (10). $\Gamma_{\ \mu\nu}^{\lambda}$ 变为 Christoffel 符号, 由度规的函数给出. 进一步考虑广义协变性给出的规范固定以及四条 Gauss 定律的约束, 我们发现剩余的引力的物理自由度恰好就是 2.

那么我们要如何构造作用量来描述引力的动力学呢? 在 $GL(4,\mathbb{R})$ 规范理论中, 我们可以存在 $\theta$ 项,
\begin{equation}
S_{\theta}=\frac{\theta}{4}\int_{M}\,\text{Tr} F\wedge F.
\end{equation}
这时 $GL(4, \mathbb{R})$ 群是非紧致的, 而且我们不知道它是否仍然是一个拓扑作用量. 可以证明 $F=0$ 为该作用量 Euler-Lagrange 运动方程的解. 对于闭的底流形 $M$, 在纤维丛约化使得  $GL(4, \mathbb{R})$ 变为  $O(4)$ 后, 这一项便会出现 Hirzebruch 符号 (为一个整数).

由于标架场同样存在, 我们可以定义一类 1--形式 $\sigma_{a}^{b}=(e^{-1})_{a}^{\mu}[D_{\nu}e_{\mu}]^{b}dx^{\nu}$ 并构造一个作用量,
\begin{equation}
\int_{M}\,\textrm{Tr}[\sigma\wedge\sigma\wedge F].
\end{equation}
虽然 $F=0$ 看起来同样是可以由该作用量的最小作用量原理自洽地给出的运动方程, 但并没有文献支持这一点.

只要式 (21) 中的运动方程可以由某种动力学实现, 规范群便将由 $GL(4, \mathbb{R})$ 约化为 $O(4)$, 且由于度规已给出, 我们可以写下无穷多个作用量. 根据一般的量纲分析, 假设作用量中导数阶次越少意味着紫外发散越少, 那么领头阶将为宇宙学常数项,
\begin{equation}
S_{\Lambda}=\Lambda M^{2}_{\textrm{pl}}\int_{M}\,e^{a}\wedge e^{b} \wedge e^{c} \wedge e^{d} \epsilon_{abcd},
\end{equation}

而次领头阶则为 Einstein-Hilbert 作用量,
\begin{equation}
S_{EH}=M^{2}_{\textrm{pl}}\int_{M}\,e^{a}\wedge e^{b} \wedge [\bar{D}A^{A}]_{d}^{c} \eta^{de}\epsilon_{abce},
\end{equation}
这里 $M_{\textrm{pl}}$ 为 Planck 能标. 仍然需要注意这些曲率相互独立以及线性项的存在是引力中特有的现象. 在 QCD 中, 这些现象在 $\bar{D}A^{A}$ 的二阶展开中才开始出现, 所以与 QCD 相比, 引力作用量显得并不稳定. 这也解释了为何 Einstein 方程中总存在许多非静态的运动方程解. 同时, 我们可以构造物质场作用量如下:
\begin{equation}
S_{m}=\int_{M}\,d^{4}x\bar{\psi}g^{\mu\nu}\gamma_{a}e^{a}_{\mu}(\partial_{\nu}+[A_{\nu}^{A}]^{b}_{c}\eta^{cd}\gamma_{b}\gamma_{d})\psi(x),
\end{equation}
$\gamma_{a}$ 为 $4\times 4$ Dirac 矩阵.

假定其他可能的高阶导数项均可忽略, 并考虑 $S=S_{\Lambda}+S_{EH}+S_{m}$ 的最小作用量原理, 由对 $A_{\mu}^{A}$ 的变分 (这里还需假设费米场不存在非标量凝聚态) 则可以得到式 (26) 所示的无挠条件. 在此基础上, 等效原理便不再是前提假设, 而是可以仿射联络的动力学导出的. 对 $e_{\mu}^{a}$ 的变分则可以得到一般的 Einstein 方程或是广义相对论的 2--形式\footnote{由式 (25), 可以得到 Riemann 张量 $R_{\ \mu\nu\rho}^{\lambda}=[e^{-1}]^{\lambda}_{a}[\bar{D}_{\rho}A_{\nu}^{A}]_{b}^{a}e_{\mu}^{b}$.}. 我们将上述讨论归纳如表 2.
\begin{table}
\begin{center}\caption{如何由第一阶形式给出一般的广义相对论 (从上至下为逻辑顺序)}
\begin{tabular}{|c|c|c|c|c|}
  \hline
  % after \\: \hline or \cline{col1-col2} \cline{col3-col4} ...
   条件& 方程& 规范对称性&场&自由度 \\
   \hline
  标架丛 & ——& $GL(4,\mathbb{R})$ & $[A_{\mu}]^{a}_{b},[e_{\mu}]^{a}$&80\\
  丛约化(度规性) & $[D_{\nu}e_{\mu}]^{a}=0$ & $O(4)$+规范条件&$[A_{\mu}]^{a}_{b},[e_{\mu}]^{a}(g_{\mu\nu})$ &40\\
  等效原理(零挠率) & $[\bar{D}_{\nu}e_{\mu}]^{a}-[\bar{D}_{\mu}e_{\nu}]^{a}=0$ & $O(4)$+规范条件&$[e_{\mu}]^{a}(g_{\mu\nu})$&16 \\
  O(4)规范固定& 许多方式& 规范条件&$g_{\mu\nu}$ &10\\
  \hline
\end{tabular}

\end{center}
\end{table}
\section{标架丛约化与 Higgs 机制}
在这一节中\footnote{这一节在源日本文献中没有, 是英文版新增的一节.}, 我们将讨论一类特殊情况下的纤维丛约化——Higgs 机制. 在我们的引力理论中, 如果我们将标架场看做 Higgs 场[7], 那么自然地会给出运动方程 (21).

我们从 Kobayashi 与 Nomizu 的教材中的观点出发, 来归纳主丛是如何约化的. 大致有以下两个步骤.

1. 当且仅当主丛 $P(G,M)$ (这里 $G$ 为主丛的结构群, $M$ 为底流形) 的配丛 $E(G/H,M,G)$ 允许截面存在时, 主丛 $P(G,M)$ 才可约化为  $P(H,M)$, $H$ 为 $G$ 的子群.

2. $P(H,M)$ 的联络可由 $E(G/H,M,G)$ 截面的条件唯一给定, 这使得它与 $P(G,M)$ 中的联络是平行的.

这里  $E(G/H,M,G)$ 为原有主丛的配丛, 其纤维空间为商群 $G/H$.

实际上, 该主丛约化正是物理中的 Higgs 机制的体现. 第一个步骤可以解释为在商群 $G/H$ 中 Higgs 场得到一个真空期望值. 我们知道在这种情况下, 原有的 $G$ 群规范对称性将破缺至 $H$ 群的规范对称性. 而第二个条件决定了原有规范场的哪一个部分仍服从 $H$ 场的规范变换.

由文献 [12], 可以考虑一个最简单的例子来确认上述结论, 比如存在 Higgs 场的 $G=SO(3)$ 规范理论, 或等价的, 存在自伴 Higgs 场的 $SU(2)$ 规范理论. 在标准规范下, Higgs 场会被赋予一个为常值的真空期望值 $h_{0}=(0,0,v)^{T}$, 但如果我们允许它做规范变换, 我们会得到与 $x$ 相关的真空期望值
\begin{equation}
h(x)=g(x)h_{0},
\end{equation}
由于 $g(x)\in SO(2)$ 保持 $h_{0}$ 不变, 这实际上决定了 $G/H$ 丛的一个截面. 从而 Kobayashi 和 Nomizu 的第一步骤得以实现.

接着让我们确定 $SO(2)$ 规范场的剩余部分, Kobayashi 与 Nomizu 的第二步告诉我们  $E(G/H,G,M)$ 的截面或 Higgs 场应当与 $G$ 规范场的原有联络平行, 即该条件可以完全由 Higgs 场的``运动方程''\footnote{此条件比一般的运动方程 $g_{\mu\nu}D_{\mu}^{SO(3)}D_{\nu}^{SO(3)}h(x)=0$ 更强.}给出,
\begin{equation}
D_{\mu}^{SO(3)}h(x)=0,
\end{equation}
这里 $D_{\mu}^{SO(3)}$ 为原有 $SO(3)$ 规范场中协变导数. 在此条件下, 原有 $SO(3)$ 规范场的两个分量被完全固定, 变为 $h$ 的函数以及剩余的 $SO(2)$ (或 $U(1)$) 规范场, 分量由下式给出
\begin{equation}
A_{\mu}^{SO(2)}=\frac{\hat{h}^{T}\left[A_{\mu}+\hat{h}\times \partial_{\mu}\hat{h}\right]}{\hat{h}^{T}\hat{h}},
\end{equation}
其中, $A_{\mu}$ 为原有 $SO(3)$ 规范场, $\hat{h}=h/v$ 为 Higgs 的无量纲表示.

对 $\hat{h}(x)=(0,0,1)^{T}$ 这最简单的情形, $A_{\mu}^{SO(2)}$ 是 $A_{\mu}$ 的第三分量, 另外两个分量则将变为零 (W 玻色子不会在经典理论中被激发). 如果 Higgs 场存在奇点 $h(x_{0})=0$, 那么 $x_{0}$ 附近的二维球面 $S^{2}$ 构成映射: $S^{2}\rightarrow SO(3)/SO(2)=S^{2}$, 该奇点由整数标记, 给出的正是 't Hooft-Polyakov 磁单极子的磁荷.

通过这种方式, 主丛约化在物理上即可解释为 Higgs 机制. 现在让我们回到引力理论中, 讨论在 Higss 机制的观点下标架丛约化是如何实现的.

不难发现 $SO(3)$ 理论中 Higgs 场运动方程 (33) 与标架场的运动方程 (21) 是相似的. 进一步说, 标架场可以写做 $e_{\mu}^{a}(x)=[g'(x)\bar{e}_{\mu}]^{a}$, 其中 $\bar{e}$ 为恒定的固定背景场, 而 $g'(x)$ 为 $GL(4,\mathbb{R})$ 的规范变换, 度规则为
\begin{equation}
g_{\mu\nu}=[g'\bar{e}_{\mu}]^{a}[g'\bar{e}_{\nu}]^{b}\eta_{ab},
\end{equation}
它决定了 $E(G/H,G,M)$ 丛上的截面, 这里 $G=GL(4, \mathbb{R})$, $H=O(4)$. 注意到如果  $g'\in O(4)$, 度规将保持不变, 即标架场通过 Higgs 机制将 $GL(4, \mathbb{R})$ 标架丛约化至 $O(4)$ 标架丛, 度规便是 Higgs 场的真空期望值\footnote{这并不是一个新观点, 已有文献指出过.}. 式 (21) 恰恰可以实现这点.

在 $G\rightarrow H$ 对应的一般 Higgs 机制中, 不难构造一个 $G-$ 不变的能够自发产生真空期望值的 Higgs 作用量. 但正如我们之前所述, 对于引力, 想构造这样的一个作用量就比较困难了, 因为引力理论中我们无法利用度规, 它会破缺原有的  $GL(4, \mathbb{R})$ 对称. 同时  $GL(4, \mathbb{R})$ 的非紧致性也同样是量子理论构造中的一大难题.
\section{量子引力为何遥不可及?}
最后让我们回到最初的问题——为何引力的量子化如此遥不可及? 由之前所述, 仅仅是在经典层级, 引力与其他规范场论就已经呈现出极大的不同. 更不用说涉及到量子化的问题. 经典层面的引力理论最大的不同之处便是其标架丛是可平行化的, 这导致理论中出现了标架场. 由于这篇文章 的重点在于为理解引力量子化的困难提供一些线索, 所以我们不会过多涉及以往量子化引力的各种尝试.

我们从承认 Nakanishi 的陈述 (见文献 [12]) 开始. Nakanishi 认为将度规 $g_{\mu\nu}$ 量子化并不是一个明智的选择. 就像我们之前提到的, 从纤维丛的图景来看, 度规并不是一个基本场而是复合的标架场. 就像 QCD 中的 $\pi$ 介子. 我们知道 $\pi$ 介子的有效理论——即手征微扰理论——
就是不可重整的, 但我们并不需要担心 QCD 本身的可重整性.

标架场显然是一个具有自旋 1 的矢量场, 而自旋为 1 的粒子比更高自旋的粒子更好处理也是不言自明的. 在一般的教科书 [13] 中, 我们仅需处理自旋 0, 1/2 与 1 的粒子. 而以标架场表示的式 (29) 中的宇宙学常数项以及式 (30) 中的 Einstein-Hilbert 作用量看起来却与通常由度规表示的作用量有些差别, 它们看起来几乎是不可能被重整化的. 但至少, 由于这些项和作用量中并没有任何系数存在负质量量纲, 我们可以将由标架场表示的作用量考虑为标架场的 4 点自相互作用, 而由度规表示的则可以看做规范场的 3 点相互作用. 这时读者会发现, 如果我们恰当地引入一项标架场的动能项以及规范联络, 看起来便足以对它们构造一个可重整量子场论.

然而, 前方的路途远比我们想象的遥远. 上述情形与我们在 QCD 中添入带电荷的矢量场 (如 $\rho$ 介子) 是类似的. 而众所周知的是, 由于纵模产生的紫外发散, 仅仅引入载荷矢量场并不足以保证我们能构造一个可重整场论. 只有当此额外的载荷矢量场同时也是服从某种规范对称的规范场时 (如 $SU(2)$ 理论中的弱作用玻色子, 其质量由 Higgs 机制赋予), 紫外发散才能消除. 该事实告诉我们, 如果无法找到一类可使标架场变为其规范联络的规范对称性的话, 量子化引力于我们终归是遥不可及的.

实际上, 在以往的文献中, 将标架场当做规范场处理从而构建自洽理论的尝试浩如烟海. 其中 3 维引力是一个非常成功的例子. Witten 证明了它是可重整的 [14], 其中的 3 维标架场起着规范玻色子的作用. 在 3 维引力中, 我们可以将标架场当做不同于一般坐标变换的局域平移生成元来处理. 同时其作用量具有 Chern-Simons 作用量的形式, 所以在新的平移对称性下作用量是保持不变的. 也就是说我们扩充了规范对称性, 在该规范对称性下标架场可以视为一类规范场, 那么理论便是可重整的.

但在 4 维或是更高维数的引力中情况却不是如此. Einstein-Hilbert 型作用量在上述局部平移下无法保持不变. 这其中也有深刻的数学原因: 如第 2 节所述, 结构群必须线性地作用于纤维. 而局部平移对纤维的作用是非线性的, 这就导致局部平移无法被当做结构群. 因此需要一个比纤维丛更普适的数学工具来描述可重整的引力理论.

另外, 弄清楚表 2 中哪一步对应理论的基本原理, 在从 1--形式约化至 2--形式时哪一步给出动力学, 也同样是一些非平凡的问题. 对于表中下半部分比如无挠条件, 我们可以很自然地把它们当做理论中动力学的结果, 但却很难想象上半部分的条件是如何处理为动力学的, 更不用说 Higgs 机制的实现. 即使对于非紧致的 $GL(4,\mathbb{R})$ 群, 这些条件也仍然存在许多问题亟待解决.
\section{结论}
引力量子化的困难就在于描述它的基本数学对象——标架丛是可平行的. 可平行性引入了标架场以及一般的规范场. 标架场又带来一些在 QCD 中罕见的物理观测量, 比如挠率, 仿射联络, 同时产生不同种类的规范不变作用量. 特别是对于 Einstein-Hilbert 量, 由于其曲率是线性的, 总可以由两个标架场与一个曲率场的缩并生成.

标架场为自旋 1 的粒子可以看做是一种规范粒子, 这解释了为什么重整化对于引力是比较困难的. 但三维引力是个例外. 在 3 维引力中的标架场\footnote{译注:原文为 dreibein, 区别于之前提到的 4 维标架场 (vierbein), 因为在 3 维中只存在 3 个标架}可以看做某规范场的局部平移. 然而, 这大概只是由于其作用量恰好是 Chern-Simons 项所导致的巧合. 总之, 在数学上引力理论存在一个很大的困难: 局域平移无法与纤维丛结构群兼容.

以上便是我们的结论. 作为一个引力领域的门外汉, 作者的结论可能显得 too young to simple, 也许以标架丛为出发点量子化引力的目标从一开始就是完全错误的. 也许在一些专家看来, 之前的陈述也都是非常平凡的结论. 还有可能这篇文章中其实还存在一些更基本的问题. 但不管怎样, 作者明白了 QCD 中绝不会出现能够对应标架场的存在, 仅是这一点就已经令他很满足了.

感谢 Kin-ya Oda 让作者将这篇文章写下来的建议, 感谢 Shigeki Sugimoto 关于标架场的物理意义的讨论, 感谢 Norihiro Tanahashi 以一位引力专家的身份对这篇文章的检阅, 感谢 Akinori Tanaka 关于为何 3 维球面是可平行化的问题的指导, 同时也感谢 Satoshi Yamaguchi 在阅读数学教材时对作者的帮助.

\begin{thebibliography}{99}
\bibitem{01}
Hidenori Fukaya, ``Why is quantum gravity so difficult (compared to QCD)?", Soryushiron Kenkyu Vol 25(2016) No. 2.
\bibitem{02}
R. Utiyama, ``Introduction to general gauge field theories" (1987) Iwatani (ISBN-10:
4000050400).
\bibitem{03}
C. Nash and S. Sen, ``Topology and Geometry for Physicists", Dover Books on Mathe-
matics, (2011) (ISBN-10: 0486478521).
\bibitem{04}
Shishichi Kobayashi and Katsumi Nomizu, Foundations of Differential Geometry Volume I, II, John Wiley \& Sons (1996) (ISBN-10: 0471157333, 0471157325).
\bibitem{05}
A. Palatini, Rend. Circ. Mat. Palermo 43, (1919) 203.
\bibitem{06}
F. W. Hehl, J. D. McCrea, E. W. Mielke and Y. Ne'eman, Phys. Rept. 258, 1 (1995) doi: 10.1016/0370-1573(94)00111-F [gr-qc/9402012].
\bibitem{07}
G. Sardanashvily, Int. J. Geom. Meth. Mod. Phys. 13, no. 06, 1650086 (2016) doi: 10.1142/S0219887816500869 [arXiv: 1602.06776 [math-ph]].
\bibitem{08}
M. Heller, ``Evolution of Space--Time Structure", Concepts of Physics 3, 2006, pp. 119-133.
\bibitem{09}
Mikio Nakahara, Geometry, Topology and Physics CRC Press, (2003) (ISBN-10: 0750306068).
\bibitem{10}
M. Luscher, Nucl. Phys. B 549, 295 (1999) doi: 10.1016/S0550-3213(99)00115-7 [hep-lat/9811032].
\bibitem{11}
M. Honda, Prog. Theor. Phys. 63, 1429 (1980). doi: 10.1143/PTP. 63. 1429.
\bibitem{12}
Noboru Nakanihi, ``Quantum gravity and general relativity", Soryushiron Kenkyu Vol. 1 (2009).
\bibitem{13}
M. Srednicki, ``Quantum Field Theory", Cambridge University Press (2007) (ISBN-10: 0521864496).
\bibitem{14}
E. Witten, Nucl. Phys. B 311, 46 (1988) doi: 10.1016/0550-3213(88)90143-5.
\end{thebibliography}
\end{document}
